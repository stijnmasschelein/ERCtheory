\documentclass{article}
\usepackage[utf8]{inputenc}
\usepackage{natbib}

\title{Theory Should I Stay or Should I Go}
\author{Stijn Masschelein}
\date{November 2018}

\begin{document}

\maketitle

\section{Introduction}

We model a two period game with buyer and sellers. The buyers offer a contract with price, $P_i$, and a potential control system in each period, $i$. We explain the control system later. The seller decides on how much effort $e_i$ to put in to the job after they accept the contract. The buyer's profit is given by $\pi_{Bi} = Se_i - P$ while the seller's profit is given by $\pi_{Si} = P - C e_i^2$ \footnote{This is not exactly what we did in the experiment but it is pretty close and in line with most simple models.}.

The social preferences are based on the $\alpha$-model in \citep{Bolton2008a}. That is there are two types of sellers: fair sellers who minimize the difference between between the seller's profit and the buyer's profit and egocentric sellers who maximize their own profit. We assume that for each sellers there is probability $\alpha$ that the seller is fair and a probability $1-\alpha$ that the seller is egocentric. As a start, we assume that this proportion, $\alpha$, is common knowledge.

The setting differs crucially from the setting in \citep{Choi2014}. In our setting, the buyer and the seller can decide to have a different contract in the second period after observing the first period behavior. In addition, when there are multiple sellers, the buyers can decide to switch to another seller in the second period. Vice versa, when there are multiple buyers, the sellers can decide to accept a contract with another buyer in the second period. In other words, our setting allows for competition to have another effect than purely a market price effect as in  \citep{Choi2014}. Our setting allows for competition in the form of \emph{switching behavior}.

\section{Monopsony}

As a benchmark case, we assume that there is only buyer and one seller. In the last period, the fair seller will split the surplus equally between buyer and seller, and the egocentric seller will exert minimum effort.

\subsection{No control}

\section{Seller competition}

Second, we assume there are two sellers and one buyer. This means that there is competition between the sellers. This gives the buyer the advantage that they know that even when the seller in period one acts unfair, they can still switch to the other seller.  



\bibliographystyle{agsm}
\bibliography{references.bib}
\end{document}
